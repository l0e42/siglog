
% v2-acmsmall-sample.tex, dated March 6 2012
% This is a sample file for ACM small trim journals
%
% Compilation using 'acmsmall.cls' - version 1.3 (March 2012), Aptara Inc.
% (c) 2010 Association for Computing Machinery (ACM)
%
% Questions/Suggestions/Feedback should be addressed to => "acmtexsupport@aptaracorp.com".
% Users can also go through the FAQs available on the journal's submission webpage.
%
% Steps to compile: latex, bibtex, latex latex
%
% For tracking purposes => this is v1.3 - March 2012
\documentclass[prodmode,acmtecs]{acmsmall} % Aptara syntax
\usepackage[spanish,polish]{babel}
\usepackage[T1]{fontenc}
\usepackage{fancyvrb}
\usepackage{graphicx,hyperref}
\newcommand\cutout[1]{}


\usepackage[table]{xcolor}
\usepackage[utf8]{inputenc}
\usepackage[parfill]{parskip}
\usepackage{tabulary}
\PassOptionsToPackage{hyphens}{url}
\usepackage{hyperref}    
\usepackage[capitalize]{cleveref}


% Metadata Information
% !!! TODO: SET THESE VALUES !!!
\acmVolume{0}
\acmNumber{0}
\acmArticle{CFP}
\acmYear{0}
\acmMonth{0}

\newcounter{colstart}
\setcounter{page}{4}

\RecustomVerbatimCommand{\VerbatimInput}{VerbatimInput}%
{
%fontsize=\footnotesize,
fontfamily=\rmdefault
}


\newcommand{\UnderscoreCommands}{%\do\verbatiminput%
\do\citeNP \do\citeA \do\citeANP \do\citeN \do\shortcite%
\do\shortciteNP \do\shortciteA \do\shortciteANP \do\shortciteN%
\do\citeyear \do\citeyearNP%
}

\usepackage[strings]{underscore}



% Document starts
\begin{document}


\setcounter{colstart}{\thepage}

\acmArticle{CFP}
\title{{\huge\sc SIGLOG Monthly 270}

 February 2026}\author{ELLI ANASTASIADI\affil{Aalborg University, SE}\vspace*{-2.6cm}\begin{flushright}\includegraphics[width=30mm]{elli_anastasiadi.png}\end{flushright}}\begin{abstract}February 2026 edition of SIGLOG Monthly, featuring deadlines, calls and community announcements.
\end{abstract}


\maketitlee

\href{https://lics.siglog.org/newsletters/}{Past Issues}
 - 
\href{https://lics.siglog.org/newsletters/inst.html}{How to submit an announcement}
\section{Table of Contents}\begin{itemize}\item DEADLINES (\cref{deadlines}) 
 
\item CALLS 
 
\begin{itemize}\item GAIW 2026 (CALL FOR PAPERS) (\cref{GAIW2026})
\item ICGT 2026 (CALL FOR PAPERS) (\cref{ICGT2026})
\item AiML 2026 (CALL FOR PAPERS) (\cref{AiML2026})
\item LSFA 2026 (CALL FOR PAPERS ) (\cref{LSFA2026})
\item MFCS 2026 (CALL FOR PAPERS) (\cref{MFCS2026})
\item C.A.R.L.A. 2026 (CALL FOR PAPERS) (\cref{CARLA2026})
\item ITRS 2026 (CALL FOR PAPERS) (\cref{ITRS2026})
\end{itemize} 
\item JOB ANNOUNCEMENTS 
 
\begin{itemize}\item POSTDOC AND PHD POSITIONS (\cref{POSTDOCANDPHDPOSITIONS})
\end{itemize} 
\end{itemize}\section{Deadlines}\label{deadlines}\rowcolors{1}{white}{gray!25}\begin{tabulary}{\linewidth}{LL}GAIW 2026:  & Feb 04, 2026 (Submission Deadline) \\
WoLLIC 2026:  & Feb 16, 2026 (Abstracts deadline), Feb 22, 2026 (Full papers deadline) \\
ICGT 2026:  & Feb 16, 2026 (Abstracts), Feb 23, 2026 (Papers) \\
AiML 2026:  & Feb 20, 2026 (Full papers abstracts), Feb 27, 2026 (Full papers) \\
PLS15:  & Mar 30, 2026 (Abstracts deadline), May 29, 2026 (Poster abstracts deadline) \\
LSFA 2026:  & Mar 30, 2026 (Abstract deadline), Apr 04, 2026 (Full paper deadline) \\
NMR 2026:  & Apr 03, 2026 (Paper registration), Apr 10, 2026 (Paper) \\
CONCUR 2026:  & Apr 20, 2026 (Abstracts), Apr 27, 2026 (Submissions) \\
MFCS 2026:  & Apr 24, 2026 (Submissions) \\
SCML 2026:  & Apr 27, 2026 (Deadline for extended abstracts) \\
C.A.R.L.A. 2026:  & May 02, 2026 (Paper registration), May 08, 2026 (Paper  deadline) \\
ITRS 2026:  & May 15, 2026 (Paper) \\
\end{tabulary}
\section{GAIW 2026: The 8th Games, Agents, and Incentives Workshop}\label{GAIW2026}  Paphos, Cyprus, May 26, 2026\\ 
  Webpage: \href{https://gtep-workshops.github.io/gaiw2026/}{https://gtep-workshops.github.io/gaiw2026/}\\ 
  co-located with AAMAS 2026\\ 
CALL FOR PAPERS 

\begin{itemize}\item  IMPORTANT DATES 
 
\rowcolors{1}{white}{gray!25}\begin{tabulary}{\linewidth}{LL}Submission Deadline:  & February 4, 2026 (AoE) \\
Acceptance Notification:  & Mar 20, 2026 \\
Camera Ready:  & TBD (AoE) \\
Workshop:  & May 26, 2026 \\
\end{tabulary}
 
\item  ABOUT  
 
  We invite submissions to the 8th iteration of the Games, Agents and Incentives Workshop, co-located with AAMAS 2026 in Paphos, Cyprus, \href{https://cyprusconferences.org/aamas2026/}{https://cyprusconferences.org/aamas2026/}. Games, Agents and Incentives is a confederated workshop which focuses on agents and incentives in AI.  In particular, it promotes approaches that deal with game theory (cooperative and non-cooperative), social choice, and agent-mediated e-commerce aspects of AI systems. The confederated workshop merges multiple workshops that have been associated with AAMAS in the past, which considered different aspects of the general interplay between AI and economics including CoopMAS, AMEC, and EXPLORE. 
 
  Over the past two decades, the focus of agent incentives in decentralised and centralised AI systems has increased dramatically. These issues come up when designing preference aggregation mechanisms and markets; computing equilibria and bidding strategies; facilitating cooperation among agents; and fairly dividing resources. 
 
\item  SUBMISSION GUIDELINES  
 
  Authors should submit full papers electronically in PDF format at: \href{https://openreview.net/group?id=ifaamas.org/AAMAS/2026/Workshop/GAIW}{https://openreview.net/group?id=ifaamas.org/AAMAS/2026/Workshop/GAIW} Formatting Guidelines: Please format papers according to the AAMAS 2026 format (\href{https://cyprusconferences.org/aamas2026/submission-instructions/}{https://cyprusconferences.org/aamas2026/submission-instructions/}). Paper Length: Papers can be at most 8 pages long in AAMAS format. Additional pages may be used for references. Supplemental material can be appended at the end of the paper. However, reviewers are instructed to make their evaluations based on the main submission, and are not obligated to consult the supplemental material. 
 
  Multiple Submissions: To widen participation and encourage discussion, there will be no formal publication of workshop proceedings. We will, however, post the accepted papers online to the benefit of the participants to the workshop. Therefore, submission of preliminary work and papers to be submitted or in preparation for submission to other major venues in the field are encouraged. 
 
  Past Submissions: In order to strike a balance between new work and work that may have been presented, but not widely seen, we ask that if authors want to submit published work they do so non-anonymously and clearly indicate when and where the work was published. We will only accept work which has been published in the last calendar year (e.g., IJCAI 2025, NeurIPS 2025, AAAI 2025, AAAI 2026 and any conference held strictly after Jan 1, 2025). 
 
  We invite papers on topics of game theory, mechanism design, fair allocation, computational social choice, and their applications to multi-agent systems.  
 
\item  Best Presentation Award: The organizing committee of GAIW will be giving two awards (first-place and runner up) for best paper presentations. The award criteria include the clarity of presentation, the level of engagement, the content, and discussion handling. 
 
  Inquiries: If you have any questions, direct them to alan.tsang@carleton.ca. 
 
\end{itemize}\section{ICGT 2026: International Conference on Graph Transformation}\label{ICGT2026}  Rennes, France, June 29th - July 3rd\\ 
  \href{https://conf.researchr.org/track/icgt-2026/icgt-2026-icgt-research-papers}{https://conf.researchr.org/track/icgt-2026/icgt-2026-icgt-research-papers}\\ 
CALL FOR PAPERS 

\begin{itemize}\item  We invite submissions to the 19th International Conference on Graph Transformation (ICGT 2026), to be held in Rennes, France, as part of STAF 2026 (Software Technologies: Applications and Foundations) between 29th June and 3rd July. ICGT is the leading venue for research on graphs and graph transformation, encompassing theoretical foundations, tools, and applications. Graph-based formalisms are central to diverse domains including: software architectures, model driven engineering, network topologies, cyber-physical systems, quantum computing, exploring molecular structures, and graph databases. 
 
\item  ICGT aims to broaden participation 
 
  ICGT is not only a theory conference, and we also welcome applications, tools, and case studies. If your work uses graph rewriting technologies (e.g., graph databases, quantum compilers, AI/ML), ICGT is for you. 
 
\item  TOPICS  
 
\begin{itemize}\item  Theory: e.g. Models, verification, concurrency, logical aspects
\item  Applications: e.g. Software engineering, quantum computing, AI/GNNs, graph databases, business processes
\item  Tools: e.g. Languages, environments, tool support
\end{itemize} 
\item  IMPORTANT DATES (AoE) 
 
\rowcolors{1}{white}{gray!25}\begin{tabulary}{\linewidth}{LL}Abstracts:  & Feb 16, 2026 \\
Papers:  & Feb 23, 2026 \\
Notification:  & Apr 06, 2026 \\
Camera-ready:  & Apr 27, 2026 \\
\end{tabulary}
 
\item  SUBMISSION INFO 
 
  Accepted papers will appear in LNCS proceedings. The submission categories are:  
 
\begin{itemize}\item  Regular Research Papers (16 pp LNCS)
\item  Tool Presentations (8 pp LNCS)
\item  Vision Papers / Open Challenges (8 pp LNCS)
\item  Journal-First Extended Abstracts (2 pp LNCS)
\end{itemize} 
  Full call details: \href{https://conf.researchr.org/track/icgt-2026/icgt-2026-icgt-research-papers}{https://conf.researchr.org/track/icgt-2026/icgt-2026-icgt-research-papers} 
 
\item  CHAIRS  
 
 Blair Archibald and Oszkar Semerath 
 
\end{itemize}\section{AiML 2026: 16th International Conference on Advances in Modal Logic }\label{AiML2026}  29 June – 3 July 2026\\ 
  Amsterdam, The Netherlands \\ 
  \href{https://events.illc.uva.nl/aiml2026/}{https://events.illc.uva.nl/aiml2026/} \\ 
CALL FOR PAPERS 

\begin{itemize}\item  ABOUT  
 
  Advances in Modal Logic is an initiative aimed at presenting the state of the art in modal logic and its various applications. The initiative consists of a conference series together with volumes based on the conferences. AiML 2026 is organized by the Institute of Logic, Language and Computation (ILLC) of the University of Amsterdam (UvA). The conference will take place on 29 June - 3 July 2026.   
 
\item  TOPICS 
 
  We invite submissions on all aspects of modal and related logic, including (but not limited to): 
 
\begin{itemize}\item  Semantics and model theory
\item  Proof theory, also including automated deduction
\item  Applications of modal logic
\item  Co-algebraic aspects
\item  History of modal logic
\item  Philosophy of modal logic
\item  Computational or theoretical aspects
\item  Specific instances and variations of modal logic (e.g., description logics, dynamic logics, epistemic and deontic logics, modal logics for agent-based systems, provability and interpretability logics, spatial and temporal logics, hybrid logic, intuitionistic logic, substructural logics)
\end{itemize} 
\item  SUBMISSIONS  
 
  There will be two types of submissions for AiML 2026: 
 
\begin{itemize}\item  Full papers for publication in the proceedings and presentation at the conference. The proceedings will be published open access via Electronic Proceedings in Theoretical Computer Science (EPTCS). 
\item  Short presentations intended for presentation at the conference but not for the published proceedings.
\end{itemize} 
  Both types of papers should be submitted electronically using the EasyChair submission page: \href{https://easychair.org/conferences/?conf=aiml2026}{https://easychair.org/conferences/?conf=aiml2026} For more instructions on submission types please visit the website.  
 
\item  IMPORTANT DATES 
 
\rowcolors{1}{white}{gray!25}\begin{tabulary}{\linewidth}{LL}Full papers abstracts:  & Feb 20, 2026 \\
Full papers submission:  & Feb 27, 2026 \\
Full papers acceptance notification:  & Apr 24, 2026 \\
Short presentations submission:  & 5 May 2026 \\
Camera-ready version full papers:  & May 15, 2026 \\
Short presentations acceptance notification:  & May 20, 2026 \\
Registration deadline:  & t.b.a. \\
Camera-ready version short presentations:  & May 29, 2026 \\
Conference:  & Jun 29 – Jul 3, 2026 \\
\end{tabulary}
 
\item  INVITED SPEAKERS  
 
\begin{itemize}\item  Kit Fine (New York University)
\item  David Gabelaia (Razmadze Mathematical Institute)
\item  Mojtaba Mojtahedi (Ghent University)
\item  Aybüke Özgün (University of Amsterdam)
\item  Sara Uckelman (Durham University)
\end{itemize} 
\item  CONTACT  
 
  To get in touch with the organizers please write to: aiml2026-illc@uva.nl 
 
\end{itemize}\section{LSFA 2026: 21st International Symposium on Logical and Semantic Frameworks with Applications}\label{LSFA2026}  18 - 19 July 2026\\ 
  Lisbon, Portugal\\ 
  \href{https://lsfa-workshop.github.io/2026/}{https://lsfa-workshop.github.io/2026/}\\ 
CALL FOR PAPERS  

\begin{itemize}\item  ABOUT 
 
  Logical and semantic frameworks are formal languages used to represent logics, languages and systems. These frameworks provide foundations for the formal specification of systems and programming languages, supporting tool development and reasoning. 
 
  We are inviting formal submissions on the following topics, but not limited to: 
 
\begin{itemize}\item  Automated deduction
\item  Applications of logical and/or semantic frameworks
\item  Computational and logical properties of semantic frameworks
\item  Formal semantics of languages and systems
\item  Implementation of logical and/or semantic frameworks
\item  Lambda and combinatory calculi
\item  Logical aspects of computational complexity
\item  Logical frameworks
\item  Process calculi
\item  Proof theory
\item  Semantic frameworks
\item  Specification languages and meta-languages
\item  Type theory
\end{itemize} 
  The program committee is chaired by Valeria de Paiva, Topos Institute, Berkeley and Thaynara de Lima, Federal University of Goiás, Goiânia. 
 
\item  SUBMISSIONS 
 
  Contributions should be written in English and submitted in the form of: 
 
\begin{itemize}\item  full papers (with a maximum of 16 pages excluding references) or ; 
\item  short papers (with a maximum of 6 pages excluding references). 
\end{itemize} 
  They must be unpublished and not submitted simultaneously for publication elsewhere. The papers should be prepared in latex using EPTCS style. \href{https://style.eptcs.org/}{https://style.eptcs.org/} The submission should be in the form of a PDF file uploaded to HotCRP. \href{https://submissions.floc26.org/lsfa/}{https://submissions.floc26.org/lsfa/} If software or data is relevant to a paper, a link that provides access to the software/data must be provided to enable reproduction of results. 
 
  Following LSFA traditions, besides Proceedings, we are considering publishing a Special Issue LSFA 25+26 (more details regarding past publications at \href{https://lsfa-workshop.github.io/}{https://lsfa-workshop.github.io/}).  
 
\item  IMPORTANT DATES  
 
\rowcolors{1}{white}{gray!25}\begin{tabulary}{\linewidth}{LL}Abstract deadline:  & Mar 30, 2026 \\
Full paper deadline:  & Apr 04, 2026 \\
Notification of acceptance:  & May 04, 2026 \\
Conference:  & Jul 18-19, 2026 \\
\end{tabulary}
 
\end{itemize}\section{MFCS 2026: 51th conference on Mathematical Foundations of Computer Science}\label{MFCS2026}  Paris, France August 24th-28th, 2026\\ 
  \href{https://mfcs2026.irif.fr/}{https://mfcs2026.irif.fr/}\\ 
CALL FOR PAPERS 

\begin{itemize}\item  MFCS is among the conferences with the longest history in the field — the first conference in the series was held already in 1972. Traditionally, the conference moved between the Czech Republic, Poland, and Slovakia; since 2013, the conference has traveled around Europe. 
 
\item  The conference will be preceded, on August 23, by the Young Research Forum Workshop intended for students and postdocs. 
 
\item  NEW: Up to 10 papers will be accepted by the program committee, for which no presence onsite is required. 
 
\item  Important dates 
 
\rowcolors{1}{white}{gray!25}\begin{tabulary}{\linewidth}{LL}Submissions:  & Apr 24, 2026 \\
Author notification:  & Jun 19, 2026 \\
Camera-ready version:  & Jun 26, 2026 \\
YRF Workshop:  & August 23rd (afternoon) \\
Conference:  & Aug 24-28, 2026 \\
\end{tabulary}
 
\item  Deadlines are firm; late submissions will not be considered. All dates are AoE. 
 
\item  Detailed information can be found on the webpage 
 
\end{itemize}\section{C.A.R.L.A. 2026: 2nd Workshop on Cognitive Architectures for Robotics: LLMs and Logic in Action}\label{CARLA2026}  Lisbon, Portugal\\ 
  July 18, 2026\\ 
  \href{https://ws-carla.github.io/web/}{https://ws-carla.github.io/web/}\\ 
  Part of FLoC 2026, \href{https://www.floc26.org/}{https://www.floc26.org/}\\ 
CALL FOR PAPERS 

\begin{itemize}\item  IMPORTANT DATES  
 
  All dates are AoE  
 
\rowcolors{1}{white}{gray!25}\begin{tabulary}{\linewidth}{LL}Paper registration:  & May 02, 2026 \\
Paper submission deadline:  & May 08, 2026 \\
Paper notification:  & May 22, 2026 \\
Camera-Ready deadline:  & Jun 19, 2026 \\
\end{tabulary}
 
  Accepted papers will be presented as posters, with a subset selected for oral presentations. The workshop will take place in person at FLoC 2026, with virtual participation options not guaranteed at the moment. 
 
\item  GENERAL INFORMATION 
 
  The Workshop on Cognitive Architectures for Robotics: LLMs and Logic in Action (CARLA) seeks to transform the landscape of intelligent behaviors by pioneering the integration of large language models (LLMs), symbolic reasoning, and logic solvers into autonomous systems. As robotics advances toward real-world applications requiring adaptability, safety, and complex decision-making, this workshop focuses on harnessing the synergy between data-driven learning models and symbolic, logic-based systems to advance automation. This year, the workshop further expands its scope to explicitly include simulated environments as first-class experimental and methodological tools. In particular, CARLA emphasizes the use of videogames and digital twins as scalable, controllable, and safe testbeds for cognitive robotics research. These environments enable systematic investigation of embodied reasoning, LLM-driven planning, and logic-based decision-making under diverse and dynamic conditions that would be difficult or costly to reproduce in physical settings. By bridging cognitive architectures with the structured management of virtual applications, CARLA aims to foster principled approaches to transferring knowledge and behaviors learned in simulation to real-world systems, while supporting reproducibility and benchmarking across research efforts. 
 
\item  SCOPE AND SUBMISSIONS 
 
  For a detailed description of topics of interest and submission guidelines please visit the website. Submissions will be managed via the FLoC submission system. Papers will remain private during the review process. All authors must maintain up-to-date profiles to ensure proper conflict-of-interest management and paper matching. Incomplete profiles may result in desk rejection.  Submit papers through the dedicated C.A.R.L.A. submission system: \href{https://submissions.floc26.org/carla/}{https://submissions.floc26.org/carla/}  
 
  The workshop follows a single-blind review process. Submissions must not be anonymized by removing author names, affiliations, and acknowledgments.  
 
\item  ORGANIZATION 
 
\begin{itemize}\item  Fabrizio Lo Scudo, University of Calabria, Italy
\item  Denise Angilica, University of Calabria, Italy
\item  Sotirios Batsakis, Hellenic Mediterranean University, Greece
\item  Manuel Alejandro Borroto Santana, University of Calabria, Italy
\end{itemize} 
\end{itemize}\section{ITRS 2026: 12th Workshop on Intersection Types and Related Systems}\label{ITRS2026}  18th July 2026, Lisbon, Portugal\\ 
  \href{https://itrs2026.tu-dortmund.de/}{https://itrs2026.tu-dortmund.de/}\\ 
CALL FOR PAPERS 

\begin{itemize}\item  IMPORTANT DATES 
 
  All deadlines are midnight anywhere-on-earth (AoE). 
 
\rowcolors{1}{white}{gray!25}\begin{tabulary}{\linewidth}{LL}Paper submission:  & May 15, 2026 \\
Author notification:  & Jun 01, 2026 \\
Final version:  & Jun 28, 2026 \\
Workshop:  & Jul 18, 2026 \\
\end{tabulary}
 
\item  OVERVIEW 
 
  ITRS 2026 is affiliated with the 11th International Conference on Formal Structures for Computation and Deduction (FSCD 2026) and is part of the 9th Federated Logic Conference (FLoC 2026). 
 
  The ITRS workshop aims to bring together researchers working on both the theory and practical applications of systems based on intersection types and related systems. We welcome original results or surveys about ongoing research, short versions of recently published articles or papers submitted elsewhere, and surveys of ongoing work. 
 
  Possible topics for submitted papers include, but are not limited to: 
 
\begin{itemize}\item  Formal properties of systems with intersection types.
\item  Results for related systems, such as union types, refinement types, or singleton types.
\item  Applications to lambda calculus and similar systems. 
\item  Applications to pi-calculus and similar systems.
\item  Applications to programming languages and program verification.
\item  Applications to other areas, such as database query languages and program extraction from proofs.
\item  Related approaches using behavioral/intensional types to characterize computational properties.
\item  Quantitative refinements of intersection types.
\end{itemize} 
\item  SUBMISSION AND PUBLICATION 
 
  Authors are invited to submit an extended abstract (between 3 and 5 pages, excluding bibliography) in PDF format through FLoC26 submissions: \href{https://submissions.floc26.org/itrs/}{https://submissions.floc26.org/itrs/} Extended abstracts should be written in English using LaTeX. Accepted papers will be presented at the workshop and will appear on the workshop website as workshop proceedings. 
 
\end{itemize}\section{POSTDOC AND PHD POSITIONS: School of Computer Science, University of Sydney, Australia}\label{POSTDOCANDPHDPOSITIONS}JOB ANNOUNCEMENT 

\begin{itemize}\item  Postdoc positions 
 
  We have a number of postdoctoral positions available (each for 2-3 years), in all areas of Theoretical Computer Science, with at least one focusing on reactive synthesis and planning, and another on streaming algorithms. Starting date flexible, but expected around mid- or late 2026. 
 
  Base salary (Level A): approx AUD 117k–120k p.a.+ 17% superannuation (retirement plan), with funding to attend conferences, and no teaching obligations. 
 
\item  PhD positions 
 
  We also have several PhD positions available in Theoretical Computer Science, with at least one focusing on reactive synthesis and planning, one on computational geometry, one on streaming algorithms, and two on cryptography \& game theoretic mechanism design. Starting date is flexible, but expected around mid- or late 2026. 
 
  PhDs are for a duration of 3.5 years (with no mandatory coursework). The PhD scholarships include: tuition fee waiver; living allowance (approx AUD 40k/year); funding to present at international workshops and conferences; no teaching obligations, but opportunities for (remunerated) teaching available. 
 
\item  For both types of positions, excellent candidates are encouraged to contact us at sasha.rubin@sydney.edu.au or clement.canonne@sydney.edu.au. For details on the group, see \href{https://usyd-sact.github.io/index.html}{https://usyd-sact.github.io/index.html} 
 
\end{itemize}


\bigskip Links: \href{http://siglog.org/}{SIGLOG website}, \href{https://lics.siglog.org}{LICS website}, \href{https://lics.siglog.org/newsletters/}{SIGLOG Monthly}\end{document}