
% v2-acmsmall-sample.tex, dated March 6 2012
% This is a sample file for ACM small trim journals
%
% Compilation using 'acmsmall.cls' - version 1.3 (March 2012), Aptara Inc.
% (c) 2010 Association for Computing Machinery (ACM)
%
% Questions/Suggestions/Feedback should be addressed to => "acmtexsupport@aptaracorp.com".
% Users can also go through the FAQs available on the journal's submission webpage.
%
% Steps to compile: latex, bibtex, latex latex
%
% For tracking purposes => this is v1.3 - March 2012
\documentclass[prodmode,acmtecs]{acmsmall} % Aptara syntax
\usepackage[spanish,polish]{babel}
\usepackage[T1]{fontenc}
\usepackage{fancyvrb}
\usepackage{graphicx,hyperref}
\newcommand\cutout[1]{}


\usepackage[table]{xcolor}
\usepackage[utf8]{inputenc}
\usepackage[parfill]{parskip}
\usepackage{tabulary}
\PassOptionsToPackage{hyphens}{url}
\usepackage{hyperref}    
\usepackage[capitalize]{cleveref}


% Metadata Information
% !!! TODO: SET THESE VALUES !!!
\acmVolume{0}
\acmNumber{0}
\acmArticle{CFP}
\acmYear{0}
\acmMonth{0}

\newcounter{colstart}
\setcounter{page}{4}

\RecustomVerbatimCommand{\VerbatimInput}{VerbatimInput}%
{
%fontsize=\footnotesize,
fontfamily=\rmdefault
}


\newcommand{\UnderscoreCommands}{%\do\verbatiminput%
\do\citeNP \do\citeA \do\citeANP \do\citeN \do\shortcite%
\do\shortciteNP \do\shortciteA \do\shortciteANP \do\shortciteN%
\do\citeyear \do\citeyearNP%
}

\usepackage[strings]{underscore}



% Document starts
\begin{document}


\setcounter{colstart}{\thepage}

\acmArticle{CFP}
\title{{\huge\sc SIGLOG Monthly 259}

 March 2025}\author{ELLI ANASTASIADI\affil{Aalborg University, SE}\vspace*{-2.6cm}\begin{flushright}\includegraphics[width=30mm]{elli_anastasiadi.png}\end{flushright}}\begin{abstract}March 2025 edition of SIGLOG Monthly, featuring deadlines, calls and community announcements.
\end{abstract}


\maketitlee

\href{https://lics.siglog.org/newsletters/}{Past Issues}
 - 
\href{https://lics.siglog.org/newsletters/inst.html}{How to submit an announcement}
\section{Table of Contents}\begin{itemize}\item DEADLINES (\cref{deadlines}) 
 
\item CALLS 
 
\begin{itemize}\item BRIO 2025 (CALL FOR ABSTRACTS ) (\cref{BRIO2025})
\item ICLP25 (CALL FOR PAPERS) (\cref{ICLP25})
\item Commemorating Frege (CALL FOR ABSTRACTS) (\cref{CommemoratingFrege})
\item LOPSTR 2025 (CALL FOR PAPERS) (\cref{LOPSTR2025})
\item DaLi 2025 (CALL FOR PAPERS) (\cref{DaLi2025})
\item ICLP 2025 (CALL FOR WORKSHOPS) (\cref{ICLP2025})
\item FSEN 2025 (CALL FOR PARTICIPATION) (\cref{FSEN2025})
\end{itemize} 
\item JOB ANNOUNCEMENTS 
 
\begin{itemize}\item PHD in Vienna (\cref{PHDinVienna})
\item Postdoc positions in Prague (\cref{PostdocpositionsinPrague})
\end{itemize} 
\end{itemize}\section{Deadlines}\label{deadlines}\rowcolors{1}{white}{gray!25}\begin{tabulary}{\linewidth}{LL}DEON 2025:  & March 05, 2025 (Abstract deadline - extended) , Mar 13, 2025 (Paper deadline - extended) \\
SSFT 2025:  & Mar 31, 2025 (Registration of applications) \\
Postdoc positions in Prague:  & Mar 31, 2025 (Deadline for applicants) \\
CONCUR 2025:  & Apr 01, 2025 (Abstract deadline), Apr 07, 2025 (Paper deadline) \\
ICLP 2025:  & Apr 13, 2025 (Paper registration - regular papers), Apr 18, 2025 (Paper submission), Jun 15, 2025 (short papers), Apr 01, 2025 (Proposal  deadline) \\
ICLP25:  & Apr 13, 2025 (Paper registration (regular papers)), Apr 18, 2025 (Paper  (regular papers)), Jun 15, 2025 (Paper  (TC papers, IJCAI Fast Track papers)) \\
Commemorating Frege:  & Apr 30, 2025 (Abstract) \\
LOPSTR 2025:  & May 09, 2025 (Abstract), May 16, 2025 (Paper) \\
DaLi 2025:  & Jun 01, 2025 (Abstract  deadline), Jun 05, 2025 (Full paper  deadline) \\
\end{tabulary}
\section{BRIO 2025: FINAL RESEARCH MEETING   “BRIO – BIAS, RISK, OPACITY in AI”}\label{BRIO2025}  1-2 July 2025\\ 
  Department of Philosophy, University of Milan, Italy \\ 
  \href{https://sites.unimi.it/brio/final-research-meeting/}{https://sites.unimi.it/brio/final-research-meeting/}\\ 
CALL FOR ABSTRACTS  

\begin{itemize}\item  INFO  
 
  BRIO – BIAS, RISK, OPACITY in AI: Design, Verification, and Development of Trustworthy AI is a collaborative National Research Project involving the University of Milan, Politecnico di Milano, the University of Genoa, the National Research Council of Italy in Trento, and the University of Naples. The project explores the challenges and limitations of Trustworthy AI through philosophical analyses of transparency, bias, and risk, alongside their formalization and technical implementation. The project's closing event will serve as a key opportunity for interdisciplinary exchange, bringing together experts and scholars in trustworthy, explainable, ethical, and fair AI.   
 
\item  INVITED SPEAKERS  
 
\begin{itemize}\item  Emily Postan (University of Edinburgh) 
\item  Sander Beckers (Cornell University) 
\item  Edemilson Paraná (LUT University) 
\end{itemize} 
\item  INDUSTRIAL TRACK - Bridging Industry and Academia  
 
   In collaboration with MIRAI, a spinoff of the Department of Philosophy at the University of Milan, the Research Meeting will feature a dedicated Industrial Track. This session will bring together invited partners from leading companies that develop and implement AI systems across various industries. The Industrial Track aims to provide insight into how businesses are navigating the rapidly evolving AI landscape while fostering stronger collaborations between industry and academia. Join us for a unique opportunity to engage with experts at the forefront of AI innovation.  
 
\item  SUBMISSIONS 
 
  We welcome contributions in the form of abstracts. Authors of accepted papers will be invited to present their work in person in a 20-minute talk, followed by a 10-minute Q\&A session. Additionally, selected authors may be invited to submit extended versions of their papers for consideration in a special issue of a reputable journal.  
 
\item  Areas of interest include:  
 
\begin{itemize}\item  Philosophy of Science and Technology 
\item  Ethics of Technology 
\item  Logics and Formal Ontologies Applied to Technology 
\item  Foundational Analysis and Ontology-Based Modeling of Trust, Bias, and Risk 
\item  Explainable AI 
\item  Machine Learning and Deep Learning Methods for Trustworthy AI  
\end{itemize} 
\item  IMPORTANT DATES 
 
\begin{itemize}\item  dh abstracts deadline: April 1 2025 
\end{itemize} 
  The submission is done via this form \href{https://forms.gle/VXgPHA5LLLF3otsB7}{https://forms.gle/VXgPHA5LLLF3otsB7}  
 
\item  Format: Submissions should include a brief description of the work (maximum 100 words) for reviewer assignment and an abstract (up to 1000 words) plus references, submitted in PDF format.  
 
\item  Notification: The program committee will inform contributors of results by 01/05/2025  
 
\end{itemize}\section{ICLP25: 41st International Conference on Logic Programming ()}\label{ICLP25}  University of Calabria, Rende, Italy \\ 
  September 12-19, 2025\\ 
  \href{https://iclp25.demacs.unical.it/}{https://iclp25.demacs.unical.it/}\\ 
CALL FOR PAPERS 

\begin{itemize}\item  SCOPE 
 
  Since the first conference In Marseille in 1982, ICLP has been the premier international event for presenting research in logic programming. Contributions are sought in all areas of logic programming, including but not restricted to: 
 
\begin{itemize}\item  Theoretical Foundations:
\end{itemize} 
  Formal and operational semantics, Non-monotonic reasoning, Reasoning under uncertainty, Knowledge representation, Semantic issues of combining logic and neural models, Complexity results. 
 
\begin{itemize}\item  Language Design and Programming Methodologies:
\end{itemize} 
  Concurrency and parallelism, Mobility, Interacting with ML, Logic-based domain-specific languages, Hybrid logical and imperative/functional languages, Programming techniques, Theory reasoning, Answer set programming, Inductive logic programming, Coinductive logic programming. 
 
\begin{itemize}\item  Program Analysis and Optimization:
\end{itemize} 
  Analysis, Transformation, Verification, Debugging, Profiling, Visualization, Logic-based validation of generated programs. 
 
\begin{itemize}\item  Implementation Methodologies and Applications:
\end{itemize} 
  Compilation, Constraint implementation, Ethics and trustworthiness, Explainability, Parallel/distributed execution, Search and optimization problems, Heuristic methods, Logic-based prompt engineering, Tabling, User interfaces. 
 
\item  IMPORTANT DATES 
 
\rowcolors{1}{white}{gray!25}\begin{tabulary}{\linewidth}{LL}Paper registration (regular papers):  & Apr 13, 2025 \\
Paper submission (regular papers):  & Apr 18, 2025 \\
Notification (regular papers):  & May 25, 2025 \\
Paper submission (TC papers, IJCAI Fast Track papers):  & Jun 15, 2025 \\
Revision submission (TPLP papers):  & Jun 15, 2025 \\
Final notification:  & Jul 06, 2025 \\
Final version:  & Jul 27, 2025 \\
Main conference:  & Sep 15-19, 2025 \\
\end{tabulary}
 
\item  TRACKS AND SPECIAL SESSIONS 
 
  In addition to the main track, ICLP’25 will host: 
 
\begin{itemize}\item  IJCAI Fast Track: The notification date for IJCAI’25 does not allow authors of rejected papers to submit to ICLP’25. In coordination with the IJCAI’25 program chairs, we have therefore instituted a process by which authors can submit revised versions of such rejected papers directly to ICLP’25. Authors must submit a cover letter explaining how they have addressed the critical issues raised by IJCAI’25 reviewers before submitting their revised paper to the IJCAI Fast Track of ICLP’25. The submission will then enter the “revision” phase and be considered for publication in TPLP. 
\item  Recently Published Research Track: Detailed information will be announced separately.
\end{itemize} 
\item  AFFILIATED EVENTS 
 
\begin{itemize}\item  Workshops: September 12-14, 2025
\item  Autumn School in Computational Logic: September 12-14, 2025
\item  Doctoral Consortium: September 12-14, 2025
\item  Logic Programming Contest: September 16 or 17, 2025 
\item  International Symposium on Principles and Practice of Declarative Programming (PPDP 2025)
\item  International Symposium on Logic-based Program Synthesis and Transformation (LOPSTR 2025)
\end{itemize} 
\item  SUBMISSION DETAILS 
 
  All submissions must be written in English. Papers accepted at ICLP may appear either in 
 
\begin{itemize}\item    The journal Theory and Practice of Logic Programming (TPLP) published by Cambridge University Press. TPLP format is described at: \href{https://www.cambridge.org/core/journals/theory-and-practice-of-logic-programming/information/author-instructions/preparing-your-materials}{https://www.cambridge.org/core/journals/theory-and-practice-of-logic-programming/information/author-instructions/preparing-your-materials}
\item    The ICLP 2025 Technical Communication Proceedings published by Electronic Proceedings in Theoretical Computer Science (EPTCS). EPTCS format is described at: \href{http://style.eptcs.org}{http://style.eptcs.org}
\end{itemize} 
\item  Submissions may have one of two forms: 
 
\begin{itemize}\item  Technical Communication (TC) papers are at most 12 pages in EPTCS format, excluding references. Accepted TC papers will be published in the Technical Communication Proceedings.
\item  Regular papers and IJCAI Fast Track papers are at most 14 pages in TPLP format, including references. Accepted regular and IJCAI Fast Track papers will be published in a special issue of TPLP. IJCAI Fast Track papers must be accompanied by a PDF cover letter detailing:
\item  The improvements made to the paper compared to the previous (IJCAI’25) submission, including clarifications on any perceived errors in the reviewers' assessments, if applicable
\item  The paper ID of the IJCAI’25 submission
\item  The authors listed on the IJCAI’25 submission
\item  The title of the IJCAI’25 submission
\item  The original PDF submitted to IJCAI’25
\item  The IJCAI’25 reviews, including scores and text evaluations
\end{itemize} 
  The authors of IJCAI Fast Track papers must explicitly give consent for IJCAI’25 to share all submitted information with ICLP’25 to verify its accuracy. ICLP’25 may summarily reject papers for several reasons, including submissions that (a) are outside the thematic scope of ICLP, (b) inaccurately disclosed required information, or (c) omitted original authors without justification.   
 
  Regular papers that are not (provisionally) accepted for TPLP may be invited to the Technical Communication Proceedings of ICLP’25. The authors can choose to convert a regular paper accepted for the Technical Communication Proceedings into an extended abstract (2 or 3 pages in EPTCS format), which should allow for submitting a long paper version elsewhere.  
 
  Submissions will be made via EasyChair, following the link \href{https://easychair.org/conferences/?conf=iclp25}{https://easychair.org/conferences/?conf=iclp25} 
 
  All papers must describe original, previously unpublished research, and must not simultaneously be submitted for publication elsewhere. These restrictions do not apply to Recently Published Research Track submissions as well as previously accepted workshop papers with a limited audience and/or without archival proceedings. 
 
  All accepted papers will be presented during the conference. Authors of accepted papers will be automatically included in the list of ALP members, who will receive quarterly updates from the Logic Programming Newsletter at no cost. 
 
\item  VENUE  
 
  ICLP’25 will be held on the campus of the University of Calabria in Rende, Italy, during 12-19 September 2025. The University of Calabria is one of Italy's leading academic institutions, renowned for its innovative research and vibrant campus life. Located in the scenic city of Rende, it offers a modern learning environment surrounded by natural beauty and cultural richness. Calabria is a region rich in culture, offering a blend of historical heritage and stunning natural beauty. From its breathtaking coastal spots to its easily accessible mountains, the region provides an unforgettable cultural and culinary experience, savoring authentic dishes made from fresh, local ingredients, such as spicy 'nduja, pasta, potatoes and exquisite desserts. 
 
\item  ORGANIZATION 
 
\begin{itemize}\item  General Chair: Francesco Ricca
\item  Program Co-chairs: Martin Gebser and Daniela Inclezan
\item  Publicity Chairs: Manuel Borroto and Francesco Calimeri
\item  Local Chairs: Antonio Ielo and Giuseppe Mazzotta
\end{itemize} 
\end{itemize}\section{Commemorating Frege: Logic and Philosophy of Mathematics}\label{CommemoratingFrege}CALL FOR ABSTRACTS 

\begin{itemize}\item  The “Commemorating Frege: Logic and Philosophy of Mathematics” Workshop will take place in Lisbon over three days, from Wednesday 10 September through Friday 12 September 2025, at the Lisbon Academy of Sciences, under the auspices of LanCog and the Centre of Philosophy of the University of Lisbon. 
 
\item  This workshop will commemorate the centenary, in 2025, of Gottlob Frege’s death, and aims to bring together philosophers working on Frege's and Fregean approaches to logic and the philosophy of mathematics, both from a historical and a contemporary perspective.  
 
\item  Confirmed invited speakers: Francesca Boccuni (Vita Salute San-Raffaele University), Philip Ebert (University of Stirling), Leila Haaparanta (University of Helsinki), Erich H. Reck (UC Riverside), Kai F. Wehmeier (UC Irvine), and Crispin Wright (University of Stirling). 
 
\item  In addition to six invited talks, the organisers are calling for submissions for six contributed talks. We welcome  proposals, especially from early career researchers, that engage with Fregean approaches to logic and the philosophy of mathematics. Contributed talks are expected to be 60 minutes-long (plus 30 minutes-long discussion). 
 
\item  Proposals for contributed talks should consist of 300-word abstract. These should be sent by email attachment in PDF format, with name and affiliation, to bmjacinto@ciencias.ulisboa.pt and jbmillan@ciencias.ulisboa.pt by Wednesday 30 April 2025.  
 
\item IMPORTANT DATES 
 
Abstract submission: Apr 30, 2025 
 
  Notification of the talks selected for presentation will be by Friday 30 May 2025.  Talks will be given in-person. We are applying for funding that we hope will enable us to cover or at least help with travel expenses and accommodation of the contributed speakers.   
 
\item  For more information or enquiries, please contact bmjacinto@ciencias.ulisboa.pt or jbmillan@ciencias.ulisboa.pt.  
 
\end{itemize}\section{LOPSTR 2025: 35th International Symposium on Logic-Based Program Synthesis and Transformation }\label{LOPSTR2025}  September 9-10, 2025 \\ 
  University of Calabria, Rende, Italy\\ 
  \href{https://lopstr.github.io/2025/}{https://lopstr.github.io/2025/}\\ 
  Part of ICLP 2025 and co-located with PPDP 2025 \href{https://iclp25.demacs.unical.it/}{https://iclp25.demacs.unical.it/}\\ 
CALL FOR PAPERS 

\begin{itemize}\item  IMPORTANT DATES 
 
\rowcolors{1}{white}{gray!25}\begin{tabulary}{\linewidth}{LL}Abstract submission:  & May 9, 2025 (AoE) \\
Paper submission:  & May 16, 2025 (AoE) \\
Author notification:  & June 27, 2025 (AoE) \\
Camera-ready:  & July 17, 2025 (AoE) \\
Symposium:  & Sep 9-10, 2025 \\
\end{tabulary}
 
\item   OVERVIEW 
 
  The aim of the LOPSTR series is to stimulate and promote international research and collaboration on logic-based program development. LOPSTR is open to contributions to logic-based program development in any programming language paradigm. LOPSTR has a reputation for being a lively, friendly forum for presenting and discussing work in progress.  
 
\item  LOPSTR 2025 will be held at the University of Calabria, Rende, Italy. It will be co-located with ICLP 2025 and PPDP 2025. At least one of the authors of an accepted paper is expected to attend the conference and present the paper. Information about venue and travel will be available on the ICLP 2025 website. For a full list of topics of interest please visit the website.  
 
\item  Survey papers that present some aspects of the related topics from a new perspective and papers that describe experience with industrial applications and case studies are also welcome.  
 
\item  PAPER SUBMISSION 
 
  Submissions can be made in two categories: 
 
\begin{itemize}\item  Regular Papers (15 pages max.)
\item  Short Papers (8 pages max.)
\end{itemize} 
  References will NOT count towards the page limit. Additional pages may be used for appendices not intended for publication. Reviewers are not required to read the appendices, and thus papers should be intelligible without them. All submissions must be written in English. 
 
\item  Submissions must not substantially overlap with papers/tools that have been published or that are simultaneously submitted to a journal, conference, or workshop with refereed proceedings. Submissions of Regular Papers must describe original work. Work that already appeared in unpublished or informally published workshop proceedings may be submitted (please contact the PC Chairs in case of questions). 
 
\item  Submissions of Short Papers may include presentations of exciting if not fully polished research or tool demonstrations that are of academic and industrial interest. Tool demonstrations should describe the relevant system, usability, and implementation aspects of a tool. All accepted papers will be included in the conference proceedings and published by Springer as a Lecture Notes in Computer Science (LNCS) volume. 
 
\item  After the symposium, a selection of a few best papers will be invited for submission to rapid publication in the Journal of Theory and Practice of Logic Programming (TPLP). Authors of selected papers will be invited to revise and/or extend their submissions to be considered for publication. The papers submitted to TPLP will be subject to the journal's standard reviewing process. 
 
\item  SUBMISSION GUIDELINES 
 
   Authors should submit an electronic copy of the paper (written in English) in PDF, formatted in the Lecture Notes in Computer Science style. Each submission must include on its first page the paper title; authors and their affiliations; contact author's email; abstract; and three to four keywords which will be used to assist the PC in selecting appropriate reviewers for the paper. Authors should consult Springer's authors' instructions on the author's page, and use their proceedings templates, either for LaTeX (available also in Overleaf) or for Word, for the preparation of their papers. Springer encourages authors to include their ORCIDs in their papers. In addition, upon acceptance, the corresponding author of each paper, acting on behalf of all of the authors of that paper, must complete and sign a Consent-to-Publish form. The corresponding author signing the copyright form should match the corresponding author marked on the paper. Once the files have been sent to Springer, changes relating to the authorship of the papers cannot be made. 
 
\item  Page numbers (and, if possible, line numbers) should appear on the manuscript to help the reviewers in writing their report. So, for LaTeX, we recommend that authors use: \textbackslash{}pagestyle\{plain \textbackslash{}usepackage\{lineno\} \textbackslash{}linenumbers 
 
\item  Papers should be submitted via EasyChair: \href{https://easychair.org/conferences/?conf=lopstr2025}{https://easychair.org/conferences/?conf=lopstr2025} 
 
\item  PROGRAM CHAIRS 
 
\begin{itemize}\item  Santiago Escobar, Univesitat Politecnica de Valencia, Spain
\item  Laura Titolo, Code Metal, USA
\end{itemize} 
\item  PUBLICITY CHAIR 
 
\begin{itemize}\item  Manuel Borroto, University of Calabria, Italy
\end{itemize} 
\item  HISTORY 
 
  LOPSTR is a renowned symposium that has been held for more than 30 years. The first meeting was held in Manchester, UK in 1991. Information about previous symposia: \href{http://lopstr.webs.upv.es/}{http://lopstr.webs.upv.es/}. You can find the contents of past LOPSTR symposia at DBLP (\href{https://dblp.uni-trier.de/db/conf/lopstr/index.html}{https://dblp.uni-trier.de/db/conf/lopstr/index.html}) and past LNCS proceedings at Springer (\href{https://link.springer.com/conference/lopstr}{https://link.springer.com/conference/lopstr}). 
 
\end{itemize}\section{DaLi 2025: 6th Workshop on Dynamic Logic - New trends and applications}\label{DaLi2025}  Shaanxi Normal University, Xi'an, Shaanxi Province, P. R. China\\ 
  October 20-21, 2025\\ 
  \href{http://dali2025.web.ua.pt}{http://dali2025.web.ua.pt} (temporary link)  \\ 
CALL FOR PAPERS 

\begin{itemize}\item  IMPORTANT DATES 
 
\rowcolors{1}{white}{gray!25}\begin{tabulary}{\linewidth}{LL}Abstract submission deadline:  & Jun 01, 2025 \\
Full paper submission deadline:  & Jun 05, 2025 \\
Author notifications:  & Jul 15, 2025 \\
\end{tabulary}
 
\item   OVERVIEW 
 
  Building on the ideas of Floyd-Hoare logic, dynamic logic was introduced in the 70's as a formal tool for reasoning about, and verify, classic imperative programs. Over time, its aim has evolved and expanded; DL can be seen now as a general set of ideas and tools devised for representing, describing and reasoning about diverse kind of actions, including (but not limited to) frameworks tailored for specific programming problems/ paradigms (e.g., separation logics), settings for modelling new computing domains (e.g., probabilistic, continuous and quantum computation), frameworks for reasoning about information dynamics (e.g., dynamic epistemic logics) and systems for reasoning about long term information dynamics (e.g., learning theory). 
 
\item  Both its theoretical relevance and practical potential make DLs a topic of interest in a number of scientific venues, from wide-scope software engineering conferences to modal logic specific events. The aim of the DaLí 2025 workshop is to bring together, in a single place, researchers with a shared interest in the formal study of actions (from Academia to Industry and more, from Mathematics to Computer Science and beyond) to present their work, foster discussions and encourage collaborations. 
 
\item  Previous editions of DaLí took place in Brasília (2017), Porto (2019), online (2020, 2022) and Tblisi (2023). 
 
\item  Submissions are invited on the general field of dynamic logic, its variants and applications. A more detailed topic description will be found in the webpage.  
 
\item  SUBMISSIONS AND PROCEEDINGS 
 
  We solicit two categories of papers: 
 
\begin{itemize}\item  Regular papers - describing original research results, case studies, or surveys, should not exceed 15 pages (excluding bibliography of at most two pages).
\item  Short papers - describing original research results or case studies, maybe in an incubation phase, with 6 to 8 pages (excluding bibliography of at most one page).
\end{itemize} 
  Papers must be formatted according to the guidelines for Springer LNCS papers. All submissions must be original, unpublished, and not submitted concurrently for publication elsewhere. Papers can be submitted through Easychair: \href{https://easychair.org/conferences/?conf=dal2025}{https://easychair.org/conferences/?conf=dal2025} A post-proceedings volume in Springer LNCS is being arranged. A special issue of the event with extended papers will be published in Journal of Logical and Algebraic Methods in Programming (Elsevier). 
 
\item  PROGRAM COMMITTEE CHAIRS 
 
\begin{itemize}\item  Jing Wang (Shaanxi Normal University, China)
\item  Alexandre Madeira (University of Aveiro, PT)
\end{itemize} 
\item  ORGANISING COMMITTEE CHAIR 
 
\begin{itemize}\item  Lei Li (Shaanxi Normal University, China)
\end{itemize} 
\end{itemize}\section{ICLP 2025: 41st International Conference on Logic Programming }\label{ICLP2025}  University of Calabria, Rende, Italy \\ 
  September 12-19, 2025\\ 
  \href{https://iclp25.demacs.unical.it}{https://iclp25.demacs.unical.it}\\ 
CALL FOR WORKSHOPS 

\begin{itemize}\item  ICLP 2025, the 41st International Conference on Logic Programming, will be held at the University of Calabria, Rende, Italy, from September 12 to September 19, 2025. The ICLP conference series has a long standing tradition of hosting a rich set of co-located workshops. ICLP workshops provide a unique opportunity for the presentation and discussion of work that can be preliminary in nature, novel ideas, and new open problems to a wide and interested audience. 
 
\item  Co-located workshops also provide an opportunity for presenting specialized topics and opportunities for intensive discussions and project collaboration. The topics of the workshops co-located with ICLP 2025 can cover any areas related to logic programming (e.g., theory, implementation, environments, language issues, alternative paradigms, applications), including cross-disciplinary areas. However, any relevant workshop proposal will be considered.   
 
\item  The format of the workshop will be decided by the workshop organizers, but ample time should be allowed for general discussion. Workshops can vary in length, but the optimal duration will be half a day or a full day. 
 
\item  WORKSHOP PROPOSAL 
 
  Those interested in organizing a workshop at ICLP 2025 are invited to submit a workshop proposal. Proposals should be in English and about two pages in length. They should contain: 
 
\begin{itemize}\item  The title of the workshop.
\item  A brief technical description of the topics covered by the workshop.
\item  A discussion of the timeliness and relevance of the workshop.
\item  A list of some related workshops held in the last years.
\item  The estimated length of the workshop and an estimate of the number of expected attendees.
\item  The names, affiliation and contact details (email, web page, phone) of the workshop organizers together with a designated contact person.
\item  Previous experience of the workshop organizers in workshop/conference organization.
\end{itemize} 
  Proposals are expected in text or PDF format. All proposals should be submitted to the Workshop Chairs by email (see below) by April 1, 2025. 
 
\item  REVIEWING PROCESS   
 
  Each submitted proposal will be reviewed by the Workshop, Program and General Chairs. Proposals that appear well-organized and that fit the goals and scope of ICLP will be selected. The decision will be notified by email to the responsible organizer by April 13, 2025. 
 
  The definitive length of the workshop will be planned according to the number of submissions received by the different workshops. For every accepted workshop, the ICLP local organizers will prepare a meeting room. The workshops and the conference organizers will collaborate in establishing a uniform approach to produce proficient and accessible proceedings for the workshops. 
 
\item  Workshop Organizers' Tasks: 
 
\begin{itemize}\item  Producing a ``Call for Papers'' for the workshop and posting it on the Internet and other means. A web page URL should be provided by April 27, 2025, and will be published on the ICLP 2025 home page.
\item  Providing a brief description of the workshop for the conference program.
\item  Reviewing/accepting submitted papers.
\item  Scheduling workshop activities in collaboration with the local organizers and the Workshop Chair.
\item  Providing a workshop program in a format specified by the conference organizers for posting by August 15, 2025.
\item  Coordinating the preparation of the workshop proceedings according to the specifications provided by the Workshop Chair.
\end{itemize} 
\item  Location: 
 
  Workshops will be collocated with ICLP 2025 at the University of Calabria, Rende, Italy. See the ICLP 2025 web site (\href{https://iclp25.demacs.unical.it}{https://iclp25.demacs.unical.it}) for location details. 
 
\item  IMPORTANT DATES (Tentative): 
 
\rowcolors{1}{white}{gray!25}\begin{tabulary}{\linewidth}{LL}Proposal submission deadline:  & Apr 01, 2025 \\
Notification:  & Apr 13, 2025 \\
Deadline for receipt of CfP and workshop web page URL April 27, 2025:  &  \\
Tentative paper submission deadlineJune 1, 2025:  &  \\
July 13, 2025:  & Latest Deadline for acceptance notification of paper authors \\
August 15, 2025:  & Deadline for workshop program \\
\end{tabulary}
 
\item  SUBMISSIONS 
 
  Please submit your workshop proposals by email to the Workshop Chair. 
 
\item  Workshop Chair: 
 
\begin{itemize}\item  Pierangela Bruno    pierangela.bruno@unical.it
\item  Jorge Fandinno       jfandinno@unomaha.edu
\end{itemize} 
\end{itemize}\section{FSEN 2025: Eleventh International Conference on Fundamentals of Software Engineering 2025 - Theory and Practice (FSEN '25) }\label{FSEN2025}  \href{https://conf.researchr.org/home/fsen-2025}{https://conf.researchr.org/home/fsen-2025}\\ 
  Västerås, Sweden\\ 
  7,8 April 2025\\ 
CALL FOR PARTICIPATION 

\begin{itemize}\item About FSEN  
 
  Fundamentals of Software Engineering (FSEN) is an international conference that aims to bring together researchers, engineers, developers, and practitioners from academia and industry to present and discuss their research work in the area of formal methods for software engineering. Additionally, this conference seeks to facilitate the transfer of experience, adaptation of methods, and where possible, foster collaboration among different groups. The topics of interest cover all aspects of formal methods, especially those related to advancing the application of formal methods in the software industry and promoting their integration with practical engineering techniques.  
 
\item  This year, FSEN will take place in Västerås, Sweden, on 7 and 8 April 2025. The event includes four keynote talks. The preliminary conference program can be found at: \href{https://conf.researchr.org/program/fsen-2025/program-fsen-2025/}{https://conf.researchr.org/program/fsen-2025/program-fsen-2025/} 
 
\item  REGISTRATION 
 
  To register for the conference please go to the following page: \href{https://conf.researchr.org/attending/fsen-2025/Registration}{https://conf.researchr.org/attending/fsen-2025/Registration} 
 
\begin{itemize}\item  dh Registration deadline (23:59 CET): March 14th 
\end{itemize} 
\item  KEYNOTE SPEAKERS  
 
\begin{itemize}\item  Işıl Dillig, University of Texas at Austin, United States
\item  Philipp Rümmer, University of Regensburg, Germany, and Uppsala University, Sweden
\item  Alexander Serebrenik, Eindhoven University of Technology, Netherlands
\item  Marielle Stoelinga, University of Twente, Netherlands, and Radboud University, Nijmegen, Netherlands
\end{itemize} 
\item  PROGRAM CHAIRS  
 
\begin{itemize}\item  Georgiana Caltais, University of Twente, Netherlands
\item  Hossein Hojjat, Tehran Institute for Advanced Studies, Iran
\end{itemize} 
\end{itemize}\section{PHD in Vienna: PhD student position in set theory}\label{PHDinVienna}JOB ANNOUNCEMENT 

\begin{itemize}\item I am seeking a doctoral student to work at the University of Vienna under my FWF-NCN (Austrian-Polish) project, “Generic large cardinals and determinacy,” led jointly by myself in Vienna, Austria, and Grigor Sargsyan in Gdansk, Poland.  A project summary can be found here: \href{https://www.fwf.ac.at/en/research-radar/10.55776/PIN1355423}{https://www.fwf.ac.at/en/research-radar/10.55776/PIN1355423} 
 
\item  The student would most likely work on problems related to the recent work of myself and Yair Hayut on the consistency of dense ideals and applications to partition principles that have found relevance in graph colorings and homological algebra.  However, the exact direction of research is somewhat open-ended, and the student may also have the opportunity to work with Sargysan on the connections with determinacy and inner model theory. 
 
\item  The position comes with a standard FWF pre-doc salary of 2684.10 euros per month for two years, with the possibility of extension. Interested candidates should send their CV and a letter of intent to monroe.eskew@univie.ac.at.  Please feel free to email also for any clarifications regarding the position. 
 
\end{itemize}\section{Postdoc positions in Prague: Three postdoctoral positions at the Czech Academy of Sciences}\label{PostdocpositionsinPrague}JOB ANNOUNCEMENT 

\begin{itemize}\item  The Institute of Computer Science of the Czech Academy of Sciences (ICS CAS) invites applications for postdoctoral positions within the TRUST and DEZINFO projects funded by the Czech Ministry of Education, Youth and Sports. The successful candidates will work in the LogICS group within the Theoretical Computer Science Department of the ICS CAS.  
 
\item  Applicants should hold a PhD degree in Computer Science, Mathematics or another relevant field, awarded no more than 7 years prior to the start date and no later than by the start date. The first deadline for applications is 31 March 2025. Applications received after this date will be considered until the positions are filled. The preferred start date is 1 June (DEZINFO) or 1 July (TRUST) 2025, but there is some flexibility. 
 
\item IMPORTANT DATES 
 
Deadline for applicants: Mar 31, 2025 
 
\item  The contract will be for 18 months with a possibility of extension in case of excellent record. 
 
\item  More information on the positions can be found here: 
 
\begin{itemize}\item  TRUST (ref. no. 2025/5): \href{https://www.cs.cas.cz/job-offer/Postdoctoral-position-TRUST-Bilkova-2025/en}{https://www.cs.cas.cz/job-offer/Postdoctoral-position-TRUST-Bilkova-2025/en} (contact: Marta Bílková)
\item  TRUST (ref. no. 2025/6): \href{https://www.cs.cas.cz/job-offer/Postdoctoral-position-TRUST-Hanikova-2025/en}{https://www.cs.cas.cz/job-offer/Postdoctoral-position-TRUST-Hanikova-2025/en} (contact: Zuzana Haniková)
\item  DEZINFO (ref. no. 2025/9): \href{https://www.cs.cas.cz/job-offer/postdoctoral-DEZINFO-Sedlar-2025/en}{https://www.cs.cas.cz/job-offer/postdoctoral-DEZINFO-Sedlar-2025/en} (contact: Igor Sedlár)
\end{itemize} 
\item  Simultaneous applications for more than one of the advertised positions are possible and encouraged. Candidates wishing to apply for more than one of the above positions can submit a single application, indicating the reference numbers of all the positions for which they are applying. The contact persons for the advertised positions are open to enquiries. We are looking forward to the applications. 
 
\end{itemize}


\bigskip Links: \href{http://siglog.org/}{SIGLOG website}, \href{https://lics.siglog.org}{LICS website}, \href{https://lics.siglog.org/newsletters/}{SIGLOG Monthly}\end{document}