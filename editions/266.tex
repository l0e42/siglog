
% v2-acmsmall-sample.tex, dated March 6 2012
% This is a sample file for ACM small trim journals
%
% Compilation using 'acmsmall.cls' - version 1.3 (March 2012), Aptara Inc.
% (c) 2010 Association for Computing Machinery (ACM)
%
% Questions/Suggestions/Feedback should be addressed to => "acmtexsupport@aptaracorp.com".
% Users can also go through the FAQs available on the journal's submission webpage.
%
% Steps to compile: latex, bibtex, latex latex
%
% For tracking purposes => this is v1.3 - March 2012
\documentclass[prodmode,acmtecs]{acmsmall} % Aptara syntax
\usepackage[spanish,polish]{babel}
\usepackage[T1]{fontenc}
\usepackage{fancyvrb}
\usepackage{graphicx,hyperref}
\newcommand\cutout[1]{}


\usepackage[table]{xcolor}
\usepackage[utf8]{inputenc}
\usepackage[parfill]{parskip}
\usepackage{tabulary}
\PassOptionsToPackage{hyphens}{url}
\usepackage{hyperref}    
\usepackage[capitalize]{cleveref}


% Metadata Information
% !!! TODO: SET THESE VALUES !!!
\acmVolume{0}
\acmNumber{0}
\acmArticle{CFP}
\acmYear{0}
\acmMonth{0}

\newcounter{colstart}
\setcounter{page}{4}

\RecustomVerbatimCommand{\VerbatimInput}{VerbatimInput}%
{
%fontsize=\footnotesize,
fontfamily=\rmdefault
}


\newcommand{\UnderscoreCommands}{%\do\verbatiminput%
\do\citeNP \do\citeA \do\citeANP \do\citeN \do\shortcite%
\do\shortciteNP \do\shortciteA \do\shortciteANP \do\shortciteN%
\do\citeyear \do\citeyearNP%
}

\usepackage[strings]{underscore}



% Document starts
\begin{document}


\setcounter{colstart}{\thepage}

\acmArticle{CFP}
\title{{\huge\sc SIGLOG Monthly 266}

 October 2025}\author{ELLI ANASTASIADI\affil{Aalborg University, SE}\vspace*{-2.6cm}\begin{flushright}\includegraphics[width=30mm]{elli_anastasiadi.png}\end{flushright}}\begin{abstract}October 2025 edition of SIGLOG Monthly, featuring deadlines, calls and community announcements.
\end{abstract}


\maketitlee

\href{https://lics.siglog.org/newsletters/}{Past Issues}
 - 
\href{https://lics.siglog.org/newsletters/inst.html}{How to submit an announcement}
\section{Table of Contents}\begin{itemize}\item DEADLINES (\cref{deadlines}) 
 
\item CALLS 
 
\begin{itemize}\item FoSSaCS 2026 (CALL FOR PAPERS) (\cref{FoSSaCS2026})
\item WoLLIC 2026 (CALL FOR PAPERS) (\cref{WoLLIC2026})
\end{itemize} 
\end{itemize}\section{Deadlines}\label{deadlines}\rowcolors{1}{white}{gray!25}\begin{tabulary}{\linewidth}{LL}FoSSaCS 2026:  & Oct 16, 2025 (Submission deadline) \\
FM 2026:  & Nov 25, 2025 (Abstract Submission), Dec 02, 2025 (Full Paper Submission) \\
FLOPS 2026:  & Dec 08, 2025 (Abstracts due), Dec 15, 2025 (Submission deadline) \\
WoLLIC 2026:  & Feb 16, 2026 (Abstracts deadline), Feb 22, 2026 (Full papers deadline) \\
\end{tabulary}
\section{FoSSaCS 2026: 29TH INTERNATIONAL CONFERENCE ON FOUNDATIONS OF SOFTWARE SCIENCE AND COMPUTATION STRUCTURES}\label{FoSSaCS2026}  Part of the ETAPS International Joint Conferences On Theory and Practice of Software\\ 
  \href{https://etaps.org/2026/conferences/fossacs/}{https://etaps.org/2026/conferences/fossacs/}\\ 
  April 11-16, 2026\\ 
  Torino, Italy\\ 
CALL FOR PAPERS 

\begin{itemize}\item  GENERAL 
 
  FoSSaCS seeks original papers on foundational research with a clear significance for software science. The conference invites submissions on theories and methods to support the analysis, integration, synthesis, transformation, and verification of programs and software systems. 
 
\item  TOPICS 
 
  For a complete list of topics please visit the website.  
 
\item  PROCEEDINGS 
 
  The FoSSaCS 2026 proceedings will be published in Springer LNCS as gold open access. In addition, there will be a special issue of the diamond open-access journal Logical Methods in Computer Science with extended version of selected papers from the conference. 
 
\item  SUBMISSIONS 
 
  FoSSaCS solicits just a single paper category: research papers. The page limit is 18 pages excluding references, respecting Springer’s LNCS format at the submission time. Additional material (no page limit) can be placed in a clearly marked appendix, at the end of the paper. The papers can be submitted at \href{https://easychair.org/conferences/?conf=fossacs2026}{https://easychair.org/conferences/?conf=fossacs2026} 
 
  FoSSaCS 2026 will adopt a double-blind reviewing process, in line with the other ETAPS conferences. PC members will be allowed to submit up to one paper. These submissions will be held to a higher standard -- for example, PC submissions will not be part of the final vote. 
 
\item  ARTIFACT EVALUATION 
 
  FoSSaCS 2026 will have a post-paper-acceptance voluntary artifact evaluation. Authors will be encouraged to submit artifacts for evaluation after paper notification. The outcome will not alter the paper’s acceptance decision. Guillermo Alberto Perez (University of Antwerp) will act as the FoSSaCS artifact evaluation chair. Detailed information will appear at \href{https://etaps.org/2026/conferences/ae-esop-fase-fossacs}{https://etaps.org/2026/conferences/ae-esop-fase-fossacs} 
 
\item  IMPORTANT DATES 
 
\rowcolors{1}{white}{gray!25}\begin{tabulary}{\linewidth}{LL}Submission deadline:  & Oct 16, 2025 \\
Rebuttal:  & Dec 08, 2025 \\
Paper notification:  & Dec 22, 2025 \\
Voluntary artifact submission deadline:  & Jan 08, 2026 \\
Paper final version:  & Jan 22, 2026 \\
Artifact notification:  & Feb 12, 2026 \\
Main Conference:  & Apr 13–16, 2026 \\
\end{tabulary}
 
\item  PC CHAIRS 
 
\begin{itemize}\item  Nathalie Bertrand (Inria Rennes, France)
\item  Stefan Milius (Friedrich-Alexander Universität Erlangen-Nürnberg, Germany)
\end{itemize} 
\end{itemize}\section{WoLLIC 2026: 32nd Workshop on Logic, Language, Information and Computation}\label{WoLLIC2026}  3-6 August 2026, Lima, Peru\\ 
  Universidad de Ingenieria y Tecnologia, Lima, Peru\\ 
  Center of Informatics, Federal University of Pernambuco, Brazil\\ 
  \href{https://wollic.org/wollic2026}{https://wollic.org/wollic2026} [wollic.org]\\ 
CALL FOR PAPERS 

\begin{itemize}\item  WoLLIC is an annual international forum on interdisciplinary research involving formal logic, computing and programming theory, and natural language and reasoning. Each meeting includes invited talks and tutorials as well as contributed papers. The thirty-second WoLLIC will be held at Universidad de Ingenieria y Tecnologia, Lima, Peru, August 3 to 6 2026. 
 
\item  SCOPE 
 
  Contributions are invited on all pertinent subjects, with particular interest in cross-disciplinary topics. Typical but not exclusive areas of interest are: non-classical logics; foundations of computing, programming and Artificial Intelligence (AI); novel computation models and paradigms; broad notions of proof and belief; proof mining, type theory, effective learnability and explainable AI; formal methods in software and hardware development; logical approach to natural language and reasoning; logics of programs, actions and resources; foundational aspects of information organization, search, flow, sharing and protection; foundations of mathematics; philosophical logic; philosophy of language. 
 
\item  PAPER SUBMISSION 
 
  Proposed contributions should be in English, and consist of a scholarly exposition accessible to the non-specialist, including motivation, background, and comparison with related works. Articles should be written in the LaTeX  format of LNCS by Springer (see author's instructions at \href{https://www.springer.com/gp/computer-science/lncs/conference-proceedings-guidelines}{https://www.springer.com/gp/computer-science/lncs/conference-proceedings-guidelines} They must not exceed 12 pages, with up to 5 additional pages for references and technical appendices. The paper's main results must not be published or submitted for publication in refereed venues, including journals and other scientific meetings. It is expected that each accepted paper be presented at the meeting by one of its authors in person. (At least one author is required to pay a full, on-site registration fee before granting that the paper will be published in the proceedings.) Papers must be submitted electronically at the WoLLIC 2026 EasyChair website \href{https://easychair.org/conferences/?conf=wollic2026}{https://easychair.org/conferences/?conf=wollic2026} [easychair.org]. 
 
\item  PROCEEDINGS 
 
  The proceedings of WoLLIC 2026, including both invited and contributed papers, will be published in advance of the meeting as a volume in Springer's LNCS series. In addition, abstracts will be published in the Conference Report section of the Logic Journal of the IGPL, and selected contributions will be published (after a new round of reviewing) as a special post-conference WoLLIC 2026 issue of a scientific journal (tba). 
 
\item  INVITED SPEAKERS 
 
  TBA 
 
\item  IMPORTANT DATES 
 
\rowcolors{1}{white}{gray!25}\begin{tabulary}{\linewidth}{LL}Abstracts deadline:  & Feb 16, 2026 \\
Full papers deadline:  & Feb 22, 2026 \\
Author notification:  & May 05, 2026 \\
Camera-ready version:  & May 20, 2026 \\
Workshop dates:  & Aug 3-6, 2026 \\
\end{tabulary}
 
\item  ORGANISING COMMITTEE 
 
  Ernesto Cuadro-Vargas (local chair), Anjolina de Oliveira (UFPE), Ruy de Queiroz (UFPE) 
 
\end{itemize}


\bigskip Links: \href{http://siglog.org/}{SIGLOG website}, \href{https://lics.siglog.org}{LICS website}, \href{https://lics.siglog.org/newsletters/}{SIGLOG Monthly}\end{document}